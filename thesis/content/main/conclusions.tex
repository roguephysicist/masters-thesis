\chapter{Final Remarks}\label{chap_conc}
\minitoc
\section{Conclusion}
I was able to learn and successfully apply the XP2SHG/SFG technique to a series of nanoparticles that were unfortunately not of sufficient optical quality to yield meaningful results. This was accomplished by using established methods (with the help of very capable people) to study samples that were not properly characterized. The obvious solution to this would have been to use other nanoparticles, but it was very much beyond my possibilities to obtain other samples within the time constraints imposed by my visit to the Femtosecond Spectroscopy group.

My suggestions for a revisionary work are the following.

\begin{description}
\item[Better quality samples.] The scattering problem can be completely eliminated with samples that are in good physical condition.
\item[Well characterized samples.] The purpose of this work was to characterize the nanoparticles via nonlinear spectroscopy. However, these measurements work much better if applied in conjunction with previous studies of the samples, such as TEM scans, linear measurements, etc.
\item[Apply the XP2SFG technique to metallic nanoparticles.] There are few references available on sum frequency studies involving metallic nanoparticles, especially in the two beam configuration. I think that using proper samples with a NOPA in the XP2SFG configuration would provide excellent characterization of the samples and interesting results.
\end{description}

\section{Final Remarks}
I think that every work of experimental science has its fair share of setbacks, complications, and difficulties. Sometimes the work itself can be very difficult or even dangerous. Other times, the work is so cutting edge that problems have to be solved as they come without the help of literature. Regardless of the scope of the work, \emph{all} experimentation is very touch-and-go business -- you arm yourself with the best tools available for the job and hope for the best. This work had its share of complications and setbacks, chief amongst these was the constant breakdown of lasers in both countries. Then, the poor quality of the samples which only came to light after they were in place and ready to be measured. Lastly, the lack of information about the samples did not allow for the systematic study needed to get the most out of this project.

Fortunately, Stephen Jay Gould once said that, ``Honorable errors do not count as failures in science, but as seeds for progress in the quintessential activity of correction.'' With that in mind I summarize what was learned from this.

First, the XP2SHG/SFG technique is fairly unique and specialized even amongst groups that are dedicated to surface optics and nonlinear optical techniques. Learning how this technique works and how it is used will be invaluable for future work in this field. Actually having seen it in use, and then using it for myself in the company of the people who pioneered it was a rewarding and educational experience.

Second, while the results were inconclusive, the types of measurements done on these types of samples are new and unexplored. There is much work to be done with these kinds of materials and I hope that this work can serve as a starting point for other interested scientists. I have no doubt in my mind that better samples would have yielded excellent new results.

Lastly, this entire work helped broaden my knowledge of nonlinear optics in general, as well as the many experimental techniques used everyday by scientists everywhere. Even so, I only possess a very small portion of the ``big picture'' needed to understand every aspect of this work. There is still a lot to be learned about surface optics and nonlinear techniques and I hope that this work, at the very least, will pique the readers' interest on these topics.
