\chapter[Introduction]{Introduction}\label{chap_intro}
\minitoc
\section{Motivation}
The principal motivation for this work is the analysis of different nanoparticles following two concepts:

\begin{description}
\item[First,] the use of second-order nonlinear optical effects that are very effective for surface analysis.
\item[Second,] the use of a special technique for optical spectroscopy -- the two beam, cross-polarized second harmonic/sum-frequency generation (XP2SHG/ SFG) technique. 
\end{description}

Metallic nanoparticles are currently a ``hot topic'' in the scientific world because the scope of their potential applications is very large, from biological applications \cite{baroli2007penetration}, imaging and detection \cite{lindfors2004detection, haick2007chemical, berciaud2006photothermal}, and more \cite{kamyshny2005ink, krenn1999squeezing}. Silicon nanoparticles, although more common, are not far behind -- their use in new solar cell technologies \cite{pillai2007surface} and biological markers \cite{li2004water} are also cutting edge research. 

None of the techniques described in this thesis are particularly new, but they have been used with great success in a variety of different materials. Although some literature exists on metallic nanoparticles characterized by these techniques, there is a relatively small amount of research on the subject. The optical methods included here are both non-destructive and potentially surface specific. Combining these with interesting nanostructures may prove to be a promising path for future developments in the nanosciences.

The focus of this thesis will be the study of second-order nonlinear effects in nanosystems. Nanoparticles have huge surface to volume ratios because they are so small; they contain few atoms and the bulk is tiny compared to the outside surface. The second-order nonlinearities, second harmonic generation (SHG) and sum-frequency generation (SFG), are both suitable for spectroscopic analysis of nanoparticles. These can be greatly enhanced using the two beam, cross-polarized SHG/SFG (XP2SHG/SFG) technique \cite{figliozzi2005single}. The purpose of this work is to study the properties of these nanostructures using the aforementioned methods.

\section{Nonlinear Optics in a Nutshell}\label{chap_intro_nonlin}
Linear optics has long dominated the study of light. Much like Newton's mechanics, it describes an incomplete picture of the interaction between light and the matter that forms our world. This is not to say the picture is incorrect; the interactions described work for our everyday situations. We call them ``linear'' because matter interacts in a directly proportional way with the electric field of the incoming light. The linear response of most materials is more appreciable than the other responses, making them difficult to observe -- we call these ``nonlinear effects.'' We will elaborate further on this point in chapter \ref{chap_theory}.

We can approximate most potentials within the atom using a harmonic oscillator model. These potentials represent the effect electrons feel when confined. They restrict the way electrons can move and determine many of the important material properties. These can tell us whether a material would make a good semiconductor for an optical device or for a computer microprocessor, or would make a very conductive metal, amongst many other things.

An example of a harmonic oscillator is a spring with a mass on one end. The other end is fixed and unmovable. If the mass is moved a little ways away from the equilibrium point and released, the mass will begin to oscillate for some time until it eventually stops once again at equilibrium (as it is dampened by gravity or friction). However, if you pull the spring too far it can deform from all the extra force. The spring follows a linear response according to the well established SHO equations when the displacement is small. Larger displacements are unaccounted for in this model -- now we are talking about \emph{nonlinear} behavior.

So the electrons behave in a similar manner if we model our electronic potentials as SHOs. This model works well for low intensities of incoming light, when the electron is displaced only a little from the ``bottom'' of the potential well. This provides the linear response between light and matter and is the reason why linear interactions dominate our everyday life. Although the light is very intense, the radiation that does reach us is spread out over half our world. Even when focused down to a very bright point it lacks the ability to deliver energy in an organized and efficient way. So our everyday light can only give electrons a little bit of energy and they move accordingly. Even the sun cannot provide the necessary conditions to allow electrons to move significantly from the bottom of a potential well.

We had the sun and different light bulbs, and used them often for experiments. I just explained why these sources can't help us past the linear regime. So people were stuck with this problem for a long time until a new light source, the LASER, was invented. LASER is an acronym for \emph{Light Amplification by Stimulated Emission of Radiation}. One of the main characteristics of a laser is that it emits an energetic, unidirectional, coherent beam of light that can be focused to a very small spot further concentrating the energy.

This discovery revolutionized optical science. The laser was precisely what was needed to produce high energy densities that could move the electrons away from the bottom of the potential well. Experimentalists starting shooting lasers into all kinds of materials -- and just like the spring and mass, the model stopped describing the experiment and all sorts of strange things started happening.

In this way we discovered nonlinear optics. These strange effects were difficult to explain at first. A new model had to be devised and tested against the experiments. Fortunately, it was not very long before one was created and found to work; not only did it explain everything observed until then, but it also predicted many things that had not yet been discovered \cite{boyd2003nonlinear, diels2006ultrashort, shen1984principles}. I will elaborate on the math of this new model in chapter \ref{chap_theory}.

SHG, a special case of SFG, was one of the first observed, and predominant optical nonlinearities that can appear from many substances. While all materials are technically nonlinear, the response of most are not appreciable for low intensities of incoming light, and are destroyed before we can see the effects. Some metals and semiconductors are excellent nonlinear materials, as are many different crystals. SHG is usually the first nonlinear effect to appear and can be the easiest to produce. As we will explain in section \ref{chap_theory_sum}, it is often attributed to surface emission which makes it an excellent tool for studying and characterizing surfaces and interfaces.

\section{Outline}
This thesis is divided into 5 chapters including this introduction. Chapter \ref{chap_theory} details the mathematics, formalism, and theory that make up our description of nonlinear optics. Chapter \ref{chap_setup} describes the materials to be characterized and the experimental setup used to study them. Chapter \ref{chap_results} consists of the experimental data and analysis, with comparisons to existing literature. Finally, chapter \ref{chap_conc} is dedicated to the final observations and remarks. The complete bibliography is located at the end of the document for easy reference.
