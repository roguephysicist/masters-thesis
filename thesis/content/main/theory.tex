\chapter{Nonlinear Optics and Nanoparticles}\label{chap_theory}
\minitoc
\section{A Review of Nonlinear Optics}
\subsection{Historical Overview}\label{chap_theory_hist}
The discovery of the optical maser by Townes \cite{PhysRev.112.1940} and the construction of the laser by Maiman in the late 1950s and early 1960s ushered a new age of optical discoveries. The ability to produce optical beams with these devices automatically lead to very highly focused energies distributed over very small areas. These concentrated energies allowed scientists to finally move into the optical nonlinear regime for many different materials.

The optical maser allowed for the first recorded observation of optical SHG by Franken et al. in 1961 \cite{PhysRevLett.7.118}. They produced a second beam of light at twice the frequency of the original by exciting a piece of crystalline quartz. This frequency doubling effect was dubbed SHG and was observed to be much less intense than the exciting beam.

There is a humorous anecdote about this experiment. Apparently, the editor of Physical Review Letters thought that the second harmonic dot on the photographic plate was a speck of dust, which he edited out. The image found in the article has an arrow pointing at the empty spot where it should be. However, this did not detract from the importance of the find.

Other developments followed promptly. In 1962, Bloembergen et al. \cite{PhysRev.127.1918, PhysRev.128.606} developed the mathematical framework to explain nonlinear optical phenomena. That same year, Terhune et al. \cite{PhysRevLett.8.404} observed SHG in calcite. These discoveries were amongst others \cite{lax1962nonlinear} that lead to further research into the geometrical dependence of nonlinear effects, and helped verify that the majority of the SHG signal produced in a centrosymmetric material comes from surface contribution, where inversion symmetry is broken.

In the late 1960s, Bloembergen \cite{PhysRev.174.813} and others \cite{PhysRev.178.1218} studied SHG in a variety of centrosymmetric materials and semiconductors. The advent of pulsed lasers during the 1970s \cite{nla.cat-vn2583352} allowed for even greater intensities to be obtained. Dye lasers came to prominence during these years, offering very large bandwidths and relatively short picosecond pulses. However, these lasers were very difficult to maintain and the dyes used were typically very toxic and presented serious health risks.

Interest began to form around using SHG to study surfaces and interfaces, since it had been proven \cite{PhysRevLett.46.145} to be exclusive to the surface area of a centrosymmetric material in the dipole approximation. Shen et al. published \cite{PhysRevB.38.7985} that there is also a quadrupole bulk contribution for this kind of material, and in 1989 \cite{shen89nature} published a review article summarizing most of the trends in surface spectroscopy using SHG. Theoretical work also played an important role in the 1990s, with new theoretical models by Sipe \cite{PhysRevB.53.10751} and others \cite{PhysRevB.53.4999, PhysRevB.60.14334, PhysRevB.55.2489, PhysRevB.57.2569}. Downer et al. \cite{downer2001optical} and L\"upke \cite{Lupke199975} both produced very thorough and referenced texts on SHG surface spectroscopy of semiconductors in the late 1990s and early 2000s. This period of time provided the foundations for surface optics today.

At around the same time, the first Ti:sapphire lasers were being produced and analyzed \cite{Moulton:86}. These early ultrafast lasers were capable of producing femtosecond pulses via mode-locked oscillators. Since the active medium is in solid state form, they present none of the risks of using dyes. These lasers were considerably more compact than dye lasers since they no longer needed external dye control systems. These lasers became commercial in the early 1990s.

Chirped pulse amplification (CPA) was invented in 1985 by Mourou and Strickland \cite{Strickland1985447}. This technique allowed Ti:sapphire lasers to achieve much higher peak energy without compromising the ultrashort pulse duration. During the 1990s, CPA became the prominent method for increasing energy output in Ti:sapphire lasers. At this point, Ti:sapphire lasers using the CPA technique were both compact, efficient, and cost effective. These factors would only improve over the following decade as the Ti:sapphire laser became the standard for high energy, ultrashort pulse applications.

\subsection{Defintion of Nonlinear Optics}
As explained briefly in section \ref{chap_intro_nonlin}, linear optics predominate in our everyday lives. The intensity of the light sources that surround us is typically not sufficient to modify the optical properties of a material. The discovery of the laser gave us access to higher intensity of polarized, directional, and coherent light. Beyond this, the ultrafast pulsed laser provides energy distributed into a much shorter time-frame which increases the peak irradiance delivered. These advances have greatly reduced the cost and effort needed to study nonlinear phenomena.

Light is nothing more than electromagnetic radiation, and is therefore composed of electromagnetic fields. This means that the study of how matter interacts with light is merely the study of how the light fields interact with the structure of matter. This can be readily appreciated for crystals and materials with very organized structures -- in fact, the best nonlinear materials are almost always crystalline in nature.

\subsubsection{Nonlinear Polarization and Susceptibility}
So what happens when very intense light coincides on a given material? Let us talk about the dipole moment per unit volume, or polarization $\mathbf{P}(t)$. This polarization describes the effect light has on a material and vice versa; it represents the optical response of a material.Taking Maxwell's equations with the usual considerations of zero charge density ($\rho=0$) and no free currents ($\mathbf{J}=0$), we have

\begin{align}
\nabla\cdot\mathbf{D} &= 0,\label{eq_max_1}\\
\nabla\cdot\mu_{0}\mathbf{H} &= 0,\\
\nabla\times\mathbf{E} &= -\mu_{0}\frac{\partial\mathbf{H}}{\partial t},\\
\nabla\times\mathbf{H} &= \frac{\partial\mathbf{D}}{\partial t}.
\end{align}

We take into account the nonlinearity of the material by relating the \textbf{D} and \textbf{E} fields with the total (linear and nonlinear) polarization \textbf{P},

\begin{equation}
\mathbf{D} = \epsilon_{0}\mathbf{E} + \mathbf{P}.
\end{equation}

Proceeding in the usual manner for deriving the wave equation, we obtain
\begin{equation}
\nabla\times\nabla\times\mathbf{E} + \frac{1}{c^{2}}\frac{\partial^{2}}{\partial t^{2}}\mathbf{E} = -\frac{1}{\epsilon_{0}c^{2}}\frac{\partial^{2}\mathbf{P}}{\partial t^{2}},
\end{equation}

which can be considerably simplified thanks to the identity
\begin{equation}
\nabla\times\nabla\times\mathbf{E} = \nabla\left(\nabla\cdot\mathbf{E}\right)-\nabla^{2}\mathbf{E}.
\end{equation}

The $\nabla\left(\nabla\cdot\mathbf{E}\right)$ term is usually negligible (for instance, if $\mathbf{E}$ is of the form of a transverse, infinite plane wave), so we can finally express the inhomogenous wave equation as
\begin{equation}
\nabla^{2}\mathbf{E} - \frac{1}{c^{2}}\frac{\partial^{2}}{\partial t^{2}}\mathbf{E} = \frac{1}{\epsilon_{0}c^{2}}\frac{\partial^{2}\mathbf{P}}{\partial t^{2}}.
\end{equation}

In this form, it is clear that the polarization acts as a source for this differential equation and we can recall our oscillator example from section \ref{chap_intro_nonlin}. The polarization can be expressed by a power series of the form
\begin{align}
{P}(t) &= \epsilon_{0}\left[\chi^{(1)}{E}(t) + \chi^{(2)}{E}^{2}(t) + \chi^{(3)}{E}^{3}(t) + \ldots\right]\label{eq_power}\\
			 &\equiv {P}^{(1)}(t) + {P}^{(2)}(t) + {P}^{(3)}(t) + \ldots,\label{eq_p_series}
\end{align}

where $\chi^{(n)}$ is the n\textsuperscript{th}-order susceptibility of the material. We can define the susceptibility as a constant of proportionallity that describes the degree of polarizability a material has in terms of the strength of an incoming optical electric field. The first term
\begin{equation}
P(t) = \epsilon_{0}\chi^{(1)}E(t),
\end{equation}

is the linear term that describes most everyday interactions between light and matter. When taking into account that the incoming fields are vectorial in nature, the linear susceptibility $\chi^{(1)}$ becomes a second-rank tensor. $\chi^{(2)}$, the second-order nonlinear optical susceptibility is a third-rank tensor \cite{boyd2003nonlinear}.

The nonlinear susceptibilities are very small in nature. If $\chi^{(1)}$ is unity, $\chi^{(2)}$ is on the order of $\approx 10^{-12}\,\text{m/V}$. This explains why such high intensity fields are needed to produce nonlinear interactions -- each term in equation \eqref{eq_power} depends on a higher power of the incoming field but has a much smaller value for the corresponding susceptibility.

A more general definition of the nonlinear polarization can be found when treating the input field as a superposition of plane waves. We assume that the electric field vector is of the form 
\begin{equation}
\mathbf{E}(\mathbf{r},t) = \sum_{n}\mathbf{E}_{n}(\mathbf{r},t),
\end{equation}

where
\begin{equation}
\mathbf{E}_{n}(\mathbf{r},t) = \mathbf{E}_{n}(\mathbf{r})e^{-i\omega_{n}t} + \text{c.c.}.
\end{equation}

If we look at the form of equation \eqref{eq_p_series}, we can express the nonlinear polarization in its full form as
\begin{equation}
\mathbf{P}(\mathbf{r},t) = \sum_{n}\mathbf{P}(\omega_{n})e^{-i\omega_{n}t}.
\end{equation}

Since we are only interested in second-order effects we can define the corresponding nonlinear polarization in terms of the second order susceptibility as
\begin{equation}
P_{i}(\omega_{n} + \omega_{m}) = \epsilon_{0}\sum_{jk}\sum_{(nm)}\chi^{(2)}_{ijk}(\omega_{n}+\omega_{m};\omega_{n},\omega_{m})E_{j}(\omega_{n})E_{k}(\omega_{m}),\label{eq_nonlin_p}
\end{equation}

where the indices $ijk$ refer to the Cartesian components of the fields, and $(nm)$ notes that $n$ and $m$ can be varied while the sum $\omega_{n} + \omega_{m}$ remains fixed.

We can study the generalized case when we have two incoming fields with frequencies $\omega_{1}$ and $\omega_{2}$. We can represent this in the following form
\begin{equation}
E(t) = E_{1}e^{-i\omega_{1}t} + E_{2}e^{-i\omega_{2}t} + \text{c.c.}\label{eq_sfg_form}
\end{equation}

Assuming the form of equation \eqref{eq_power}
\begin{equation}
P^{(2)} = \epsilon_{0}\chi^{(2)}E(t)^{2},
\end{equation}

and substituting expression \eqref{eq_sfg_form} we get
\begin{align}
P^{(2)}(t) &= \epsilon_{0}\chi^{(2)}\left[E^{2}_{1}e^{-i2\omega_{1}t} + E^{2}_{2}e^{-i2\omega_{2}t}\right.\nonumber\\
&+\left. 2E_{1}E_{2}e^{-i(\omega_{1}+\omega_{2})t} + 2E_{1}E^{\ast}_{2}e^{-i(\omega_{1}-\omega_{2})t} + \text{c.c.}\right]\nonumber\\
&+ 2\epsilon_{0}\chi^{(2)}\left[E_{1}E^{\ast}_{1} + E_{2}E^{\ast}_{2}\right].\label{eq_second_order}
\end{align}

We separate this expression into its components and the nonlinear effect that each represents in the following manner (abbreviations defined in table \ref{tab_janner}),
\begin{align}
P(2\omega_{1}) &= \epsilon_{0}\chi^{(2)}E^{2}_{1}e^{-i2\omega_{1}t} + \text{c.c.}\quad\text{(SHG)},\nonumber\\
P(2\omega_{2}) &= \epsilon_{0}\chi^{(2)}E^{2}_{2}e^{-i2\omega_{2}t} + \text{c.c.}\quad\text{(SHG)},\nonumber\\
P(\omega_{1}+\omega_{2}) &= 2\epsilon_{0}\chi^{(2)}E_{1}E_{2}e^{-i(\omega_{1}+\omega_{2})t} + \text{c.c.}\quad\text{(SFG)},\label{eq_list}\\
P(\omega_{1}-\omega_{2}) &= 2\epsilon_{0}\chi^{(2)}E_{1}E^{\ast}_{2}e^{-i(\omega_{1}-\omega_{2})t} + \text{c.c.}\quad\text{(DFG)},\nonumber\\
P(0) &= 2\epsilon_{0}\chi^{(2)}\left(E_{1}E^{\ast}_{1} + E_{2}E^{\ast}_{2}\right) + \text{c.c.}\quad\text{(OR)}.\nonumber
\end{align}

Janner \cite{janner1998exciton} has a wonderfully formatted table in her dissertation that summarizes the first few optical processes, which I reproduce here as follows.
\begin{table}[H]
\centering
\scalebox{0.9}{
\begin{tabular}{| c c c | p{6.5cm} | c |}
\hline
\multicolumn{3}{|c|}{$\quad\,\,\chi^{(n)}(-\omega;\omega_{1},\ldots,\omega_{n})$}	& Process & Order \\ \hline
$-\omega$ &;										& $\omega$				& Linear absorption / emission and refractive index									& 1 \\ \hline
$0$ & ; 											& $\omega,-\omega$		& Optical rectification (OR)															& 2 \\ \hline
$-\omega$ & ; & $0,\omega$													& Pockels effect																	& 2 \\ \hline
$-2\omega$ & ; & $\omega,\omega$											& Second-harmonic generation (SHG)													& 2 \\ \hline
$-(\omega_{1}+\omega_{2})$ & ; & $\omega_{1},\omega_{2}$					& Sum-frequency generation (SFG)													& 2 \\ \hline
$-(\omega_{1}-\omega_{2})$ & ; & $\omega_{1},\omega_{2}$					& Difference-frequency generation (DFG) / Parametric amplifcation and oscillation	& 2 \\ \hline
$-\omega$ & ; & $0,0,\omega$												& d.c. Kerr effect																	& 3 \\ \hline
$-2\omega$ & ; & $0,\omega,\omega$											& Electric Field induced SHG (EFISH)												& 3 \\ \hline
$-3\omega$ & ; & $\omega,\omega,\omega$										& Third-harmonic generation (THG)													& 3 \\ \hline
$-\omega$ & ; & $\omega,-\omega,\omega$										& Degenerate four-wave mixing (DFWM)												& 3 \\ \hline
$-\omega$ & ; & $-\omega_{2},\omega_{2},\omega_{1}$							& Two-photon absorption (TPA) / ionization / emission								& 3 \\ \hline
\end{tabular}}
\caption{Optical processes described by $\chi^{(n)}(-\omega;\omega_{1},\ldots,\omega_{n})$\label{tab_janner}}
\end{table}

From this point forward we will only be concerned with second-order effects.

\subsubsection{Symmetry Considerations for Centrosymmetric Materials}\label{chap_theory_sym}
As mentioned previously, $\chi^{(2)}$ is a third-rank tensor with 27 elements. The amount of non-zero elements varies with the symmetry properties of the medium. Knowing these properties can help us reduce the amount of unknown elements to calculate.

I will mention only one that proves to be of extreme importance for surface optics. A centrosymmetric material, or a material with an inversion center, is a material that for every point at coordinates $(x,y,z)$, there is an identical point located at $(-x,-y,-z)$. For instance, many crystals are centrosymmetric. If we assume that we are in the bulk of a centrosymmetric material, we can write the nonlinear polarization as
\begin{equation}
{P}(t) = \epsilon_{0}\chi^{(2)}{E}^{2}(t).\label{eq_regular}
\end{equation}

If the medium is centrosymmetric, a sign change must affect both the electric field and the polarization. So,
\begin{align}
-{P}(t) &= \epsilon_{0}\chi^{(2)}\left[-{E}(t)\right]^{2},\\
			  &= \epsilon_{0}\chi^{(2)}{E}^{2}(t).\label{eq_centro}
\end{align}

However, substituting \eqref{eq_centro} into \eqref{eq_regular} we get ${P}(t) = -{P}(t)$. We can finally deduce that
\begin{equation}
\chi^{(2)} = 0.
\end{equation}

Therefore, all second-order processes are forbidden in the bulk of centrosymmetric materials in the dipole approximation. We will talk about the other important approximation in section \ref{chap_theory_quad}. This property is broken at the surface since that region no longer presents an inversion center.  This very special property is what enables second-order nonlinearities to be so effective for surface and interface measurements. Likewise, any other mechanism that breaks the symmetry, such as an electric field or mechanical stress will also allow a second-order signal to be produced. See Bloembergen's \cite{bloembergen1999surface} excellent review about second-order effects for surface spectroscopy for further reading. 

\subsection{Bulk Quadrupolar and Other Contributions}\label{chap_theory_quad}
Everything that I have stated up to this point assumes what we call the \emph{dipole approximation} that arises from assuming that the polarization can take the form of a multipole expansion. The dipole approximation simply assumes that the dipolar contribution is significantly greater than all the others. This is not necessarily the case in many materials. In particular, we find that there can be a non-negligible electric quadrupole contribution from the bulk of centrosymmetric materials. Bloembergen et al. \cite{PhysRev.174.813} elaborate on this as early as the 1960s. This adds a severe complication to the use of second-order nonlinearities as surface probes since signal is actually produced from both surface and bulk. Sipe et al. \cite{sipe1987fundamental} go into some detail about this problem, stating that it is very difficult to separate the surface and bulk contributions as the various nonlinear coefficients cannot be measured separately. Guyot-Sionnest and Shen \cite{PhysRevB.38.7985} go one step further and state that the contributions are impossible to separate. They suggest that the best way to distinguish one from the other is by taking measurements before and after altering the surface and observing the overall changes to the produced signal. About a decade later, Shen et al. \cite{shen1999surface} state that bulk contributions not only come from the electric quadrupole, but also from the magnetic dipole, although the latter is typically much less intense than either of the former. They express the bulk polarization as a multipole series as follows,

\begin{equation}
\mathbf{P}^{B}(\omega) = \mathbf{P}_{D}(\omega) - \nabla\cdot\mathbf{Q}(\omega) - \left(\frac{c}{i\omega}\right)\nabla\times\mathbf{M}(\omega) + \ldots,
\end{equation}

where $\mathbf{P}_{D}(\omega)$ is the dipolar polarization, $\mathbf{Q}(\omega)$ is the electric quadrupole polarization, and $\mathbf{M}(\omega)$ is the magnetic dipole polarization. Indeed, if only the dipolar contribution is forbidden for centrosymmetric materials then there will be a contribution from the other two in addition to the dipolar contribution at the surface. The group does however go on to explain that there are a few experimental ways to help distinguish between surface and bulk contributions.

If $\mathbf{Q}(\omega)$ is assumed to take some form similar to

\begin{equation}
\mathbf{Q}(\omega) \approx \chi^{(2)}_{q}(\omega_{1} + \omega_{2})\mathbf{E}(\omega_{1})\nabla\mathbf{E}(\omega_{w}),
\end{equation}

then $\chi^{(2)}_{q}$ is a fourth-rank tensor with 81 independent elements. Clearly this adds some considerable complication to our problem and makes selecting the appropriate symmetry that much more important.

In summary, bulk electric quadrupole and magnetic dipole contributions to second-order surface effects may not be negligible and need to be taken into account. We will see later in sections \ref{chap_theory_nps} and \ref{chap_theory_xp2} that these considerations are important for studying nanoparticles when using the XP2SHG/SFG technique.

\subsection{SFG and SHG}\label{chap_theory_sum}
We call the third process in expression \eqref{eq_list} Sum-frequency generation (SFG). It is a second-order process that involves two photons, of frequencies $\omega_{1}$ and $\omega_{2}$ that combine to form one photon of frequency $\omega_{3} = \omega_{1} + \omega_{2}$. This is represented mathematically in the previous expression
\begin{equation}
P(\omega_{1}+\omega_{2}) = 2\epsilon_{0}\chi^{(2)}E_{1}E_{2}e^{-i(\omega_{1}+\omega_{2})t} + \text{c.c.},
\end{equation}

where the term is explicitly stated in the exponential. 

A special case of sum-frequency generation is when both incoming frequencies are the same, i.e. $\omega_{1} = \omega_{2}$. The resulting frequency is then exactly double that of the input frequency. 

As mentioned previously, second-order nonlinear processes are prohibited in the bulk of centrosymmetric materials (in the dipole approximation). Since it has a very strong surface contribution (where the inversion symmetry is broken), it can be used as a very precise diagnostic tool for surface and interface regions.

The use of these second-order nonlinearites for surface studies had gained momentum in the 1990s. McGilp wrote a review about using SHG and SFG as surface and interface probes in 1996 \cite{mcgilp1996review}. He adds experimental confirmation to his theories in 1999 \cite{mcgilp1999second} in an extremely thorough review about using SHG on the surface of almost any material you can think of. Aktsipetrov et al. \cite{aktsipetrov1997dc} followed a different approach by establishing what they call electric field induced second-harmonic generation, or EFISH. In this paper he elaborates how the sensitivity of SHG to surfaces can be enhanced by applying an electric field across the interface.

The theoretical side of things was further developed in a paper by Maytorena et al. \cite{PhysRevB.57.2569} discussing the formalities of SFG from surfaces by finding the exact expressions for the susceptibility based on modeling conductors and dielectrics. These models include fluid based, classical dynamics in addition to the wave equation treatment. A couple of interesting review papers by Downer et al. \cite{downer2001optical} and Scheidt et al. \cite{scheidt2004optical} exist, where they report results of SHG spectroscopies from a variety of different surfaces and interfaces including nanocrystals. These works are all predecessors for the later works we will discuss in section \ref{chap_theory_nps}.

\subsubsection{Phase-Matching}
What happens when the generated nonlinear wave propagates through a medium is that it becomes out of phase with the induced polarization after some distance. When this happens, the induced polarization will create new light out of phase with the light it created earlier and the two contributions will cancel out. This can be avoided if both frequencies of light (the fundamental and the produced second-order field) travel at the same phase velocity through the medium. Each wave with a different frequency will have a different wave-vector ($\mathbf{k}$) and wavenumber ($k$). Optimally, we would like a material such that

\begin{equation}
\Delta k = k_{1} + k_{2} - k_{3} = 0,\label{eq_phase}
\end{equation}

where $k_{1} = k_{2}$ for SHG. Equation \eqref{eq_phase} exemplifies a \emph{phase-matched} process. In practice, dispersion does not let this happen since the index of refraction of a material is almost never the same for different frequencies. There are certain materials that overcome this limitation (such as birefringent materials) that possess two indices of refraction.

Introducing equation \eqref{eq_phase} into the wave equation and solving, we can obtain the intensity profile \cite{boyd2003nonlinear} as

\begin{equation}
I(L) = \beta\,\vert P\vert^{2}\,L^{2}\text{sinc}^{2}\left(\frac{\Delta k L}{2}\right),
\end{equation}

where $L$ is the length of the material, $\Delta k$ is the phase mismatch, and $\beta$ are constants. The sinc function has a maximum at zero, so it is important to reduce the phase mismatch as much as possible. The inclusion of $L$ also indicates a relation to the material thickness. These considerations are important when selecting a nonlinear material such as a crystal -- most are sold in varying thicknesses that are optimized to work with certain frequencies.

In practice, phase-matching is usually improved through crystal orientation, selecting the right crystal thickness, and careful selection of the type of crystal being used.

\subsection{Optical Parametric Amplifiers}
We talked about how we can obtain different frequencies of light through wave mixing in section \ref{chap_theory_sum}. In practice however, it is considerably more difficult to implement a system in which we can easily create frequency addition or difference. It is no small task even with a fixed input wavelength. Most ultrafast lasers are tunable to some degree by adjusting internal components. We'll need something much more sophisticated if we want a variety of frequency choices.

An optical parametric amplifier (OPA) is a device that allows the user to obtain a wide bandwidth of wavelengths to work with, via the nonlinear processes of difference frequency generation (DFG) and optical parametric generation (OPG). Additionally, many commercial OPAs allow the user to tune the output by means of a motorized, computer-controlled interface. Some OPAs work on the basis of sum and difference frequency generation, using crystals to add and subtract the different frequencies in order to obtain the desired one.

OPG is a by-product of DFG. DFG occurs when a high frequency ($\omega_{1}$) photon is absorbed by an atom that jumps to a virtual level after being excited. It then decays producing two photons of lower frequency ($\omega_{2}$ and $\omega_{3}$). The creation of the $\omega_{2}$ photon is what we call OPG. If we instigate this process in the presence of an $\omega_{2}$ field, the same frequency ($\omega_{2}$) gets amplified at expense of the original $\omega_{1}$ photon. The $\omega_{3}$ frequency is called the idler and can be used in the same way as $\omega_{2}$ if desired. This effect is called optical parametric amplification. Therefore, we can create an OPA by creating a new frequency via OPG, and amplifying it using a crystal or other nonlinear media through optical parametric amplification. 

In practice most OPAs work like this: a high frequency, high power pump beam amplifies a lower frequency, lower power signal beam in a nonlinear crystal which is our desired $\omega_{2}$. This pump beam is usually the laser fundamental. This fixed pump beam transfers energy to produce the signal beam that is selectable via phase matching. This signal beam then feeds a second crystal to produce optical parametric amplification. We might be inclined to think that this is a form of stimulated emission similar to what happens in a laser (sans cavity). In stimulated emission, an electron drops from a higher level to a lower level due to the outside perturbing influence of an incident photon. It radiates with the exact same characteristics as the incoming field. OPA involves a transfer of energy from one photon to another (in our example, $\omega_{2}$ and $\omega_{1}$) to amplify $\omega_{2}$ while annihilating $\omega_{1}$.

\subsection{Noncollinear Optical Parametric Amplifiers}\label{chap_theory_nopa}
A noncollinear optical parametric amplifier (NOPA) replaces the $\omega_{2}$ signal with a white light super continuum. Tuning a NOPA is achieved by changing the angle between the seed and the pump beam, by changing the orientation of the crystal, or by using a delay stage to temporally overlap the fundamental with the desired frequency from the continuum.

In practice, the white light seed is typically generated from a sapphire window. The pump is normally the frequency doubled fundamental at 400 nm \cite{huber2001noncollinear, PhysRevB.84.165316}. The NOPA has a larger bandwidth than a regular OPA, and the resulting pulsewidth is dependent only on the bandwidth of the seed and not on the pulsewidth of the laser. For this reason, the NOPA has improved stability and spatial qualities. The added flexibility of the NOPA allows for different geometries to be implemented \cite{bodnar2010dual}. Gale \cite{gale1995sub} and Wilhelm et al. \cite{wilhelm1997sub} wrote some of the earliest papers refering to this type of OPA. Lee \cite{leecascaded} explains some of the formalism behind the operation of a NOPA.

\section{Second-Order Nonlinear Response of Nanoparticles}\label{chap_theory_nps}
The theory up to this point explains how second-order nonlinearities interact with matter and how they have been used for studying planar surfaces. I mentioned in chapter \ref{chap_intro} how nanoparticles have very large surface to volume ratios. The study of nanosystems with conventional optics has further motivated scientists to begin using second-order nonlinear phenomena to obtain more information from their samples.

I will briefly review some of the current models for describing the optical response for nanoparticles. These models consist of parametrizing the nonlinear response and then calculating the second-order emissions from the idealized nanosystem.

Dadap et al. \cite{dadap1999second} developed some early work in 1999, and later expanded on that in 2004 \cite{dadap2004theory}. They modeled SHG for a centrosymmetric nanosphere and concluded that SHG is produced via nonlocal excitation of the electric dipole moment and local excitation of the electric quadrupole moment. In other words, the electric-dipole can have excitation from either the electric quadrupole or the magnetic dipole, in addition to excitation provided by the incoming field. These results where verified experimentally by Shan et al. \cite{shan2006experimental} in an article from 2006, by taking angle- and polarization-resolved measurements of dye-coated polystyrene spheres. Brudny et al. \cite{brudny2000second} created a similar model that focuses on analytical expressions for the dipolar and quadrupolar second-order susceptibilities for a small dielectric sphere, and the nonlinear response for a Si sphere above a substrate.    

A more relevant treatment was published by Moch\'an et al. for an array of nanoparticles \cite{mochán2003second} that builds on their previous article \cite{brudny2000second}. This approach assumes spherical nanoparticles and should work well with the samples described in this work (see figures \ref{fig_tem_si_np} and \ref{fig_tem_gold_np}). I'll briefly review the method as follows.

\subsection{Theoretical Model}\label{mochan}
Let us assume a nanosphere centered at the origin. It has a linear response characterized in the usual way by its dielectric function $\epsilon(\omega)$. We assume the applied field, $\mathbf{E}^{\text{ex}}(\mathbf{r})$ is inhomogeneous and varies on a much larger scale than R, the radius of the nanoparticle. The dipole moment $\mathbf{p}^{(2)}$ is in some way related to $\mathbf{E}^{\text{ex}}(0)$ and $\nabla\mathbf{E}^{\text{ex}}(0)$, and is nonlocal as dictated by the symmetry of the sphere. We can express this relation as
\begin{equation}
\mathbf{p}^{(2)} = \gamma^{\rho}\mathbf{E}^{\text{ex}}(0)\nabla\cdot\mathbf{E}^{\text{ex}}(0) + \gamma^{e}\mathbf{E}^{\text{ex}}(0)\cdot\nabla\mathbf{E}^{\text{ex}}(0) + \gamma^{m}\mathbf{E}^{\text{ex}}(0)\times\left[\nabla\times\mathbf{E}^{\text{ex}}(0)\right],
\end{equation}

where $\gamma^{\rho}$, $\gamma^{e}$, and $\gamma^{m}$ are nonlinear response parameters. It is clearly nonlocal as it contains the field derivative. Likewise, \cite{brudny2000second} shows that the quadrupole moment, $\mathbf{Q}^{(2)}$, should be local, and symmetric of the form
\begin{equation}
\mathbf{Q}^{(2)} = \gamma^{q}\left(\mathbf{E}^{\text{ex}}(0)\mathbf{E}^{\text{ex}}(0) - \frac{1}{3}[\mathbf{E}^{\text{ex}}(0)]^{2}\,\mathbf{1}\right),
\end{equation}

where $\gamma^{q}$ is the parametric response parameter. Note that it is local and does not depend on the field derivative. As there is no external charge inside the sphere, $\gamma^{\rho} = 0$. A lengthy derivation is necessary to obtain the remaining response parameters and I will not include it here. From those parameters we obtain the values of $\mathbf{p}^{(2)}$ and $\mathbf{Q}^{(2)}$, with
\begin{equation}
Q^{(2)} = \gamma^{q}\left[\mathbf{E}^{\text{ex}}(0)\right]^{2}.
\end{equation}

Substituting these into the expression for the macroscopic nonlinear polarization for the entire array of spheres
\begin{equation}
\mathbf{P}^{nl} = n_{s}\mathbf{p}^{(2)} - \frac{1}{6}\nabla\cdot n_{s}\mathbf{Q}^{(2)} - \frac{1}{6}\nabla n_{s}Q^{(2)},
\end{equation}

where $n_{s}$ is the nanocrystal volume density; we can then obtain
\begin{equation}
\mathbf{P}^{nl} = \Gamma\nabla E^{2} + \Delta^{\prime}\left(\mathbf{E}\cdot\nabla\mathbf{E}\right).
\end{equation}

The first term can be neglected because it is longitudinal and does not radiate. With $\Delta^{'}\equiv n_{s}\left(\gamma^{e} - \gamma^{m} - \gamma^{q}/6\right)$, we can finally write the expression for the polarization as
\begin{equation}
\mathbf{P}^{(2)} = \Delta^{\prime}\left(\mathbf{E}\cdot\nabla\mathbf{E}\right),\label{eq_mochan_p}
\end{equation}

where $\mathbf{E}$ represents the incoming laser field. The relevance of this expression will become apparent in section \ref{chap_theory_xp2}, when we discuss the XP2SHG/SFG technique.

\subsection{Other Works}
Two relatively recent articles have studied different nonlinear effects on a variety of nanoscale gold shapes (primarily split-ring resonators) \cite{klein2007experiments, feth2008second}. The group hypothesizes that the samples show improved SHG due to local and nonlocal fields; in this case, magnetic resonances.

Zeng et al. \cite{zeng2009classical} developed a model based on classical electrodynamics to predict SHG in metallic nanostructures. Solving Maxwell's equations yields an expression for the SHG signal intensity and is surprisingly close to experimental values. However, the derivation is considerably easier due to the approximations.

For further reading, Link and El-Sayed \cite{link2003optical} offer a very extensive review of many other optical properties of nanoparticles.

\subsection{Summary}
In this section we saw how the theory indicates that the second-order response of nanoparticles depends both on the strength of the incoming field, but also on its characteristics. We also introduced the dependence on $\left(\mathbf{E}\cdot\nabla\right)\mathbf{E}$ of the nonlinear second-order polarization. With this information in hand we can determine the best technique to optimize our nonlinear signal.

Formally, we learned that the second-order nonlinearities are produced in nanoparticles thanks to local interface electric dipole contributions, plus quadrupolar contributions from the interior of the nanoparticles.

\section{Theory for the XP2SHG/SFG Technique}\label{chap_theory_xp2}
In 2003, Cattaneo and Kauranen \cite{cattaneo2003determination} published about a promising new technique involving two beams coinciding on a thin film. They proposed that the use of the second beam reduces the number of nonlinear coefficients to be determined if the polarization of the two beams is properly described beforehand. The method was simple enough -- they expressed the parallel and perpendicular fields separately as a sum of expansion coefficients that are in themselves linear combinations of susceptibility components. Changing the polarization of the control and probe beam would determine different coefficients. Adding the linear optical properties of the film and modeling with Green's-function led to the desired $\chi^{(2)}$ coefficients.

Following in 2005, Figliozzi et al. \cite{figliozzi2005single} note the importance of the aforementioned $\left(\mathbf{E}\cdot\nabla\mathbf{E}\right)$ term and relate it to the enhanced quadrupolar contribution; they go on to experimentally verify that dependence. They also discover that the two beam arrangement greatly enhances the entire SHG signal, both from the microscopic contributions of the particles as well as the bulk quadrupolar contribution from the substrate. This brings up the issue of how to discriminate between the two contributions to the SHG signal, which they manage to do by contrasting the difference in signal at different polarizations between the particles and the unimplanted glass.

In 2009 Wang et al. with some of the same people from references \cite{klein2007experiments} and \cite{feth2008second} elaborate a study on a gold thin film using a two-beam configuration \cite{wang2009surface}. They follow a similar model to that described in \cite{figliozzi2005single}, but their goal is separating the surface and bulk components of the SH field. This is done by finding material parameters due to the magnetic dipole and electric quadrupole by varying the polarization of the incoming beams. They succeed in finding which components of $\chi^{(2)}$ belong to surface only, bulk only, or both.

A very recent and thorough article \cite{PhysRevB.84.165316} by Wei et al. presents experimental evidence that supports the use of two-beam SHG with nanoparticles. They go into detail about the linear characterization of the Si nanocrystals used, which includes photoluminescence spectra, spectroscopic ellipsometry, and Raman microspectroscopy. By moving the sample across the overlapping beam region (the Z-scan technique) they were able to effectively determine which signal was produced by nanoparticles and which by bulk contributions from the substrate. They conclude with a very complete characterization of the Si nanocrystals after comparing the obtained data with that of the other conventional spectroscopies. Coincidentally, the setup described in this article is the exact one used in the experimental portion of this work, and is detailed in chapter \ref{chap_setup}.

A 2008 article \cite{wirth2008second} by Wirth et al. provides excellent review of the exact technique used in this work. It is also noteworthy in that it is the first article to refer to this technique as XP2SHG. Following on the work that we discussed in section \ref{mochan}, they establish that the polarization of the samples can be separated into two expressions,

\begin{align}
\mathbf{P}^{(2)}_{nc} &\equiv n_{b}\left(\gamma^{e}-\gamma^{m}-\frac{\gamma^{q}}{6}\right)\left(\mathbf{E}\cdot\nabla\right)\mathbf{E},\label{eq_p_nc}\\
\mathbf{P}^{(2)}_{g} &\equiv \left(\delta-\beta- 2\gamma\right)\left(\mathbf{E}\cdot\nabla\right)\mathbf{E},
\end{align}

where equation \eqref{eq_p_nc} is identical to \eqref{eq_mochan_p}. The article confirms that the XP2SHG technique enhances both the nanocrystal and glass signals, and goes on to say that the detected SHG signal is a product of the interference of both. This would account for shape of the plotted data from the results of the Z-scan presented in \cite{wirth2008second}, in \cite{PhysRevB.84.165316}, and in chapter \ref{chap_results} of this thesis.

The fields can be described by the amplitudes of the SH field from the nanocrystals $\vert\Gamma_{nc}\vert$, the glass $\Gamma_{g}$, and the phase $\Phi$ between them such that

\begin{align}
\mathbf{P}^{(2)}_{nc} &\equiv \vert\Gamma_{nc}\vert e^{i\Phi}\left(\mathbf{E}\cdot\nabla\right)\mathbf{E},\\
\mathbf{P}^{(2)}_{g} &\equiv \Gamma_{g}\left(\mathbf{E}\cdot\nabla\right)\mathbf{E}.
\end{align}

Three independent Z-scan measurements are needed to enable isolation of the nanocrystal signal and obtain the three unknowns $\vert\Gamma_{nc}\vert, \Gamma_{g}$, and $\Phi$: a glass scan, a scan with the nanocrystals at the entrance position, and with the nanocrystals at the exit position. They establish an empirical model based on the wave equations for each measurement, thus determining the intensity profile expressions in terms of the unknowns. All that is left is running the scans and plugging in the data to determine them and fully isolate the different contributions.

Studies like those included in \cite{PhysRevB.84.165316} and \cite{wirth2008second} are precisely in line with the work presented in this thesis.

\subsection{Signal enhancement with XP2SHG/SFG}
It was confirmed \cite{figliozzi2005single, sun2005quadrupolar} by Sun and Figliozzi et al. that the dipolar SHG single beam count rate scales as

\begin{equation}
N_{SHG} \sim f_{\text{rep}}I^{2}A\tau = \frac{f_{\text{rep}}\mathcal{E}^{2}}{\tau A},
\end{equation}

where $A$ is the beam spot size ($A = \pi w^{2}_{0}$), $\tau$ is the pulse duration, $\mathcal{E}$ is the pulse energy, and $f_{\text{rep}}$ is the repetition rate of the pulses. In correlation with the $\left(\mathbf{E}\cdot\nabla\right)\mathbf{E}$ dependence, that same group determined that the derivative creates an additional term for a Gaussian beam. This term comes from the quadrupolar contribution and introduces an extra factor of $A$, such that

\begin{equation}
N_{SHG} \sim \frac{f_{\text{rep}}\mathcal{E}^{2}}{\tau A^{2}}.
\end{equation}

If we use two incoming plane wave fields, we obtain from \eqref{eq_p_nc} (ignoring constants)

\begin{equation}
\mathbf{P}^{(2)}_{nc} \approx \left[(\mathbf{E}_{1}\cdot\nabla)\mathbf{E}_{2} + (\mathbf{E}_{2}\cdot\nabla)\mathbf{E}_{1}\right]e^{i(\mathbf{k}_{1} + \mathbf{k}_{2})\cdot\mathbf{r}}.
\end{equation}

We define the incoming fields as

\begin{equation}
\mathbf{E}_{i}(\mathbf{r}) = \hat{\epsilon}_{i}\mathcal{E}e^{i\mathbf{k}_{i}\cdot\mathbf{r}},
\end{equation}

where $\hat{\epsilon}_{i}$ is the polarization. For \emph{cross-polarized} beams, $\hat{\epsilon}_{1}\perp\hat{\epsilon}_{2}$; if $\hat{\epsilon}_{1} = \hat{y}$, then

\begin{equation}
\mathbf{P}^{(2)}_{nc} \approx \frac{1}{\lambda}\mathcal{E}_{1}\mathcal{E}_{2}\sin\alpha\hat{y},
\end{equation}

where $\frac{1}{\lambda} = k$. Now the signal intensity scales as 

\begin{equation}
N_{SHG} \sim \frac{f_{\text{rep}}\mathcal{E}_{1}\mathcal{E}_{2}\sin^{2}\alpha}{\lambda^{2}\tau A^{2}},
\end{equation}

where $\alpha$ is the angle between the beams. The $\frac{1}{\lambda}^{2}$ factor increases the intensity of the signal very significantly, while the $\sin^{2}\alpha$ term allows us to optimize the beam angle.

\subsection{Summary}
We used this section to discuss the fine points behind the XP2SHG/SFG technique. It offers three benefits over single beam second-order spectroscopy:

\begin{enumerate}
\item The SHG/SFG signal intensities are much higher than for single beam SHG/SFG.
\item The enhanced SHG/SFG signal allows for more elements of the nonlinear susceptibility $\chi^{(2)}$, and therefore of the second order polarization to be determined.
\item Dipole contribution from the surface can be discriminated from electric quadrupole and magnetic dipole contributions from bulk for planar materials.
\item Hybrid (dipole and quadrupole) contribution from the nanoparticles can be discriminated from electric quadrupole and magnetic dipole contributions from the substrate bulk.
\end{enumerate}
